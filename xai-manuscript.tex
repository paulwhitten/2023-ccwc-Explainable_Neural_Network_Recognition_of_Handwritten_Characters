\documentclass[conference]{IEEEtran}
%\IEEEoverridecommandlockouts
% The preceding line is only needed to identify funding in the first footnote. If that is unneeded, please comment it out.
\usepackage{cite}
\usepackage{amsmath,amssymb,amsfonts}
\usepackage{algorithmic}
\usepackage{graphicx}
\usepackage{textcomp}
\usepackage{xcolor}
\usepackage{subcaption}
\def\BibTeX{{\rm B\kern-.05em{\sc i\kern-.025em b}\kern-.08em
    T\kern-.1667em\lower.7ex\hbox{E}\kern-.125emX}}
\begin{document}

\title{
    Explainable Neural Network Recognition of Handwritten Characters
}

\author{\IEEEauthorblockN{Paul Whitten, Francis Wolff, Chris Papachristou}
\IEEEauthorblockA{\textit{Electrical, Computer, and Systems Engineering} \\
\textit{Case School of Engineering} \\
\textit{Case Western Reserve University}\\
Cleveland, OH, USA \\
pcw@case.edu, fxw12@case.edu, cap2@case.edu}
%\and
%\IEEEauthorblockN{Author 2}
%\IEEEauthorblockA{\textit{dept. name of organization (of Aff.)} \\
%\textit{name of organization (of Aff.)}\\
%City, Country \\
%email address}
%\and
%\IEEEauthorblockN{Author 3}
%\IEEEauthorblockA{\textit{dept. name of organization (of Aff.)} \\
%\textit{name of organization (of Aff.)}\\
%City, Country \\
%email address}
}

\IEEEoverridecommandlockouts
\IEEEpubid{\makebox[\columnwidth]{979-8-3503-3286-5/23/\$31.00~\copyright2023 IEEE \hfill} \hspace{\columnsep}\makebox[\columnwidth]{ }}

\maketitle

\IEEEpubidadjcol

\begin{abstract}

    %original
    % The prevalence of Artificial Intelligence (AI) in many aspects of life has
    % brought about increasing concern and desire for explainability.  This work poses
    % and outlines the implementation of an Explainable Artificial Intelligence data
    % pipeline, comprised of Neural Networks, for recognition of handwritten
    % characters in the EMNIST database. A key challenge for supervised learning and
    % explainability is ambiguous and mislabeled samples in training data.  Techniques
    % are applied to identify some of the issues with ambiguous and mislabeled data in
    % EMNIST.  This paper introduces a neural network architecture which recognizes
    % handwritten characters and produces explanations based on a character property
    % analysis.  A key concept to explainability that is introduced here is correctly
    % pruning the training set in order to improve the explainability and avoid
    % misleading training set labels. The neural network architecture is used to
    % resolve some of the training set issues by leveraging classifications and
    % explanations. Results from the EMNIST character database are provided.

    %part 2
    % The prevalence of Artificial Intelligence (AI), especially Machine Learning
    % (ML), in many aspects of life has brought about increasing concern and
    % desire for explainability.  Good work has been posed in an attempt to
    % interpret the weights of Neural Networks and provide a hint of
    % explainability.  Some methods, such as salience mapping and Grad-CAM,
    % provide a visual explanation by highlighting the portions of input that led
    % to a decision.  This work introduces a partitioning approach to
    % explainability and constructs an explainable data pipeline.  Decision making
    % is distributed among explainable properties and associated neural networks
    % to propose solutions.  The proposed property-based solutions are then
    % combined to make a global decision and provide explainable rationale for the
    % decision related to the properties that contributed to the solution.  The
    % particular explainable pipeline implemented recognizes handwritten
    % characters and produces explanations based on a character property analysis.
    % This work also approaches scalability by applying the explainable pipeline
    % to a dataset with nearly five times more classes than prior work.
    % Increasing the accuracy of the pipeline via adding unexplainable solutions.
    % A metric is introduced that assists in characterizing explainability when an
    % unexplainable component is added to the pipeline. A key challenge for
    % supervised learning and explainability is ambiguous and mislabeled samples
    % in training data. Techniques are applied to identify some of the issues with
    % ambiguous and mislabeled data in EMNIST.  A concept explored in this work
    % involves correctly pruning the training set in order to improve
    % explainability and avoid misleading training set labels.  Samples from the
    % training set that were pruned as label issues are presented back to the
    % explainable pipeline in an attempt resolve issues and relabel them.  Results
    % and explainable rationale from some pruned training set data is presented.

    % The prevalence of Artificial Intelligence (AI), especially Machine Learning
    % (ML), in many aspects of life has brought about an increasing desire for
    % explainability. Work has been posed to interpret the weights of Neural
    % Networks and provide visual hints of explainability. This work introduces a
    % partitioning approach, constructing an explainable architecture that
    % recognizes handwritten characters. Decisions are partitioned among
    % explainable properties and associated neural networks to pose local
    % solutions. Local solutions contribute to a global decision and explainable
    % rationale for the decision. The accuracy of the architecture is improved by
    % mixing explainable and unexplainable solutions and a metric is introduced
    % that assists in characterizing explainability using mixed solutions.
    % Scalability of the explainable architecture is explored with a dataset
    % containing nearly five times the classes of prior work. Ambiguous and
    % mislabeled samples in training data prove challenging. Techniques are
    % applied to identify and prune the training set in order to improve
    % explainability. Pruned samples from the training set are presented to the
    % architecture in an attempt to resolve issues. Results and explainable
    % rationale from pruned data is presented.

    The prevalence of Artificial Intelligence (AI), in many aspects of life, has
    brought about an increasing need for explainability of AI solutions.
    Previous work has been posed to interpret the weights of Neural Networks and
    provide visual hints of explainability. This work introduces a partitioning
    approach, constructing an explainable architecture that recognizes
    handwritten characters. The accuracy of the architecture is improved by
    introducing unexplainable components and a metric to characterize
    explainability. Ambiguous and mislabeled samples in training data prove
    challenging. Techniques are applied to identify issues and prune the
    training set to improve explainability. Pruned samples from the
    training set are presented to the architecture in an attempt to resolve
    issues. Results and explainable rationale from pruned data is presented.

\end{abstract}

\begin{IEEEkeywords}
    Explainable Artificial Intelligence, XAI, Neural Network, Machine Learning, Training Set Pruning
\end{IEEEkeywords}

\section{Introduction}

Artificial Intelligence (AI), especially Machine Learning (ML), has been used
widely in recent years.  Many aspects of human interaction with technology may
involve AI, e.g, digital assistants, online shopping, and autonomous vehicles.

While modern ML systems have been very successful in many tasks, there has been
a lack of an ability to explain decisions made by a ML system. ML systems
largely act as an opaque box.  The activity that occurs within the box is hidden
to humans.  When AI is used in making important decisions there is a strong
desire to explain why a decision was made.  Some proposed legislation suggests
that an explanation, suitable for a human, must be provided in decisions made by
automated systems.  Explainable Artificial Intelligence (XAI) has been a keen
area of interest to DARPA\cite{Gunning_Aha_2019} as well.
Section~\ref{related_work} discusses related work on ML and XAI.

% Handwritten sample datasets, such as the Modified National Institute of
% Standards and Technology database (MNIST)\cite{deng2012mnist}, which consists of
% 70,000 samples of handwritten digits, and the Extended MNIST database\cite{cohen2017emnist}
% (EMNIST), which contains over 800,000 samples, serve as
% important datasets that have been utilized widely in ML to evaluate models. The
% datasets are valuable testbeds to leverage for research. Despite the wide use of
% MNIST and EMNIST, there are problems with samples that are ambiguous or
% mislabeled.  These ambiguous and mislabeled samples are especially challenging
% for explainability.

A property-based explainable architecture is introduced in Section~\ref{method}
that provides rationale for classification decisions, referencing explainable
properties.  In this methodology a partitioning approach is taken where,
explainable properties, transformations of data related to the explainable
properties, and ML models trained against those transformations are used to pose
local solutions.  The local, property-based, solutions are then combined to make
a global decision.  The properties used in the global decision contribute
rationale for the decision.  This methodology achieved acceptable accuracy
results with MNIST and was able to provide explanations related to explainable
properties of the samples.  Results from the explainable architecture using
MNIST are presented in Section~\ref{mnist_results}.

This work explores scaling issues as it applies the explainable architecture to
the expansive EMNIST dataset, with five times more classes than prior work.
Results of processing EMNIST data is reviewed in Section~\ref{emnist_results}.
Finally, explainability examples are presented in
Section~\ref{results:exp}.

In addition to scaling issues with EMNIST, this work explores issues with
training data from ambiguous and mislabeled samples.  Examples of issues with
training data is briefly discussed in Section~\ref{related_work}. Pruning
techniques for mitigating training data issues are outlined in
Section~\ref{method:pruning}. Section~\ref{results:resolving} discusses the
results of resolving some training issues using the explainable architecture to
produce label corrections with explainable rationale for those corrections.

%% motivation
%\subsection{Motivation}

Current Neural Networks (NN) and ML models do not explain or justify decisions
in a manner suitable to present to a human user.  There is currently no
algorithm that can take weights of NN and justify decisions to a human.

This work introduces and constructs an
explainable architecture with the aim of recognizing handwritten characters while
providing a human understandable explanation for decisions.

A key challenge in explainability is resolving ambiguous and mislabeled samples
in the expanded dataset.  Experiences and challenges of utilizing the expanded
character dataset EMNIST are outlined.  This work attempts to prune some
training label issues to achieve better results.  There is a desire to resolve
label issues  and provide an explanation for the resolution.

\section{Related Work}
\label{related_work}

\subsection{Neural Networks}

Since the presentation of the McCulloch-Pitts neuron
model\cite{McCulloch1943-MCCALC-5} in 1943 and the subsequent implementation of
the perceptron\cite{rosenblatt1957perceptron}, the hardware, neural network
models, and tools have improved drastically, e.g.,
backpropagation\cite{6795724}, Convolutional Neural Networks
(CNN)\cite{fukushima1982neocognitron}, gradient based learning\cite{726791},
deep residual learning\cite{7780459}.

\subsection{Explainable Artificial Intelligence}

The ability to map the learning classifier or recognizer to human-based
explainability is a challenging task for human understandability.  Currently,
there are at least seventeen explainable techniques, such as decision tree-based,
rule-based (i.e. knowledgebase), salience mapping, Grad-CAM, sensitivity-based analysis,
feature importance, fuzzy-based, neural-network, and genetic-programming based.
These techniques use one of three basic evaluation approaches:
application-grounded, human-grounded, and functionally grounded
\cite{Arrieta2020ExplainableAI,Survey18,Fuzzy19,Hagras18,GP18,selvaraju2017grad}.

There has been work done by Vilone and Longo \cite{vilone2020explainable} to
further characterize XAI techniques.  The XAI methodology from \cite{whitten21}
takes a knowledge-based approach to construct a classification system with
intrinsic explainability.  The method involves manually discovering multiple
properties across classes of input that contribute to explainability.
Transforms related to properties were identified and applied to alter inputs.
Granular property based inferences are used to support a global decision.
Output from the property-based methodology is a textual justification for the
decision based upon the explainable properties.

\subsection{Datasets}

Handwritten sample datasets, such as the Modified National Institute of
Standards and Technology database (MNIST)\cite{deng2012mnist}, which consists of
70,000 samples of handwritten digits, and the Extended MNIST database\cite{cohen2017emnist}
(EMNIST), which contains over 800,000 samples, serve as
important datasets that have been utilized widely in ML to evaluate models. The
datasets are valuable testbeds to leverage for research. Despite the wide use of
MNIST and EMNIST, there are problems with samples that are ambiguous or
mislabeled.  These ambiguous and mislabeled samples are especially challenging
for explainability.

\subsection{Pruning Training Sets}

Data sets that have noise, such as problems with labels and ambiguous samples,
are problematic to effective ML training and explainability. Automatic pruning
using weak learners to prune noisy data sets and obtained good results in
\cite{angelova05}.  The excellent work on Confident Learning\cite{northcutt2021}
produced an open source framework, Cleanlab, to identify label issues in
datasets.  This research utilizes Cleanlab in some of the pruning techniques.

%% Method
\section{Method}
\label{method}

\subsection{Property-Based XAI Methodology}

This methodology consists of a set of steps to achieve an XAI system via image
processing, supervised learning, and probabilistic techniques. The system takes
samples for classification as input and outputs a set of pairs (class prediction
and confidence of that class) as well as rationale for the classification
decisions.

The system constructed is an explainable architecture, as depicted in
Fig.~\ref{fig:xai_data_pipeline}.  The system takes a partitioning approach to
decision-making and explainability based on explainable properties.  Explainable
properties are characteristics of input data that can be leveraged for
recognition and explainability.  Input is acted upon in the first stage to
transform it into multiple samples based upon explainable properties. The next
step performs property inferences, where solutions are posed based on each
property. The third stage, the voting mechanism, considers the results of the
multiple property inferences to make a global decision. Finally, the
explainability stage utilizes the properties that contributed to the decision to
provide explainable rationale.

\begin{figure}
    %\centering
    \includegraphics[width=9.2cm]{./images/xai-pipeline.png}
    \caption{Property-Based Explainable Architecture}
    \label{fig:xai_data_pipeline}
\end{figure}

% The methodology for building the XAI system takes training and test sets as
% input. The training and test sets are composed of sets $X$ and $Y$ where $x_j
% \in X$ and $y_j \in Y$ represent the sample and label, for the $j^{th}$ item of
% the sets.  The sample, $x_j$, from the MNIST and EMNIST datasets
% are simply a 28x28 grayscale image and the label, $y_j$, an unsigned 8 bit integer,
% representing the class of the corresponding sample. 

The first step of constructing the explainable architecture is to discover explainable properties.
Samples from the training set are manually reviewed to identify explainable
properties. Explainable properties are characteristics of particular classes of
input samples.

An example of an explainable property comes from an observation that uppercase
characters such as A, E, F, H, I and L often contain lines. Lines are an
explainable geometric property of multiple character classes. Another
explainable property comes from observing circles or semicircles in letters like
O, Q, B, P, D. A third property is called stroke.  The stroke represents the
undirected path traced by the writing implement when writing the character.

%The width can vary between characters based on the writer, tip of the implement, or pressure.

From the list of explainable properties, property transforms are identified.  A
property transform is a modification of the input to highlight an explainable
property.  Transform modules implement the conversion of the input to exemplify
incidence of the property in the input.

In the case of the line and circle explainable properties, a transform that
highlights lines and circles in images is the Hough
transform\cite{Hough1959qva}.  Modules are composed to perform a Hough
transform for linear, circle, and ellipse features.  The stroke of a character
can be highlighted by taking the morphological skeleton\cite{LEE1994} of the
input image.

After implementing transform modules for each of the properties, test and
training sets must be transformed and prepared for training. NNs are implemented
in a 1:1 relationship with transform modules and trained using the transformed
training set.  A NN is trained on, and receives input from, one property
transformation.  Each resulting trained NN will make inferences based on its
explainable property.

Upon completing the training of the NNs, the Knowledgebase (KB) is constructed.
This is done by processing the transformed training data by the trained NNs and
storing the results in the KB along with training labels.  The data in KB is
then used to gauge the effectiveness of each property transform and NN for a
particular class.  The effectiveness metrics are stored in the Knowledgbase and
are later used to assist in the voting mechanism.

A probabilistic voting scheme is constructed as a module to combine the granular
solutions and utilize the effectiveness metrics to make a set of
global classification decisions and confidence for each solution based on
effectiveness.

Finally, an XAI module is constructed that takes the output of Property
Inferencing and the global decision from voting to compose an explanation.
This XAI module relates the explainable properties that contributed to a global
decision and bases the explanation on the explainable properties.

Python applications implement the stages of the explainable architecture from
Fig.~\ref{fig:xai_data_pipeline}. Modules transform the EMNIST data according to
explainable properties, construct and train the ML models, build the
knowledgebase, implement probabilistic voting, and finally a module
constructs the rationale based on properties contributing to the global
decision.

Properties and corresponding property transforms that were utilized consist of:
\begin{itemize}
    \item The stroke property and skeleton transform
    \item The circle property and Hough circle transform
    \item The circle property and multiple non-overlapping Hough circle transforms
    \item The crossing property and the corresponding crossing transform
    \item The ellipse property and the Hough ellipse transform
    \item The endpoint property and the endpoint transform
    \item The enclosed region property and the flood fill transform
    \item The line property and the linear Hough transform
    \item The enclosed region of the stroke and the flood fill of the skeleton
    \item The area property and the convex hull transform
\end{itemize}

A new property added to this methodology is the area property, representing the
general region of the image that a character consumes. The corresponding
property transform is the convex hull. Fig.~\ref{fig:zero_raw} represents an
original zero sample character and Fig.~\ref{fig:zero_ch} represents the convex
hull transform of the original zero.

\begin{figure}[h]
    \centering
    \begin{subfigure}{.5\columnwidth}
        \centering
        \includegraphics[width=.50\textwidth]{./images/raw_0-0-12.png}
        \caption{A zero base image}
        \label{fig:zero_raw}
    \end{subfigure}%
    \begin{subfigure}{.5\columnwidth}
        \centering
        \includegraphics[width=.50\textwidth]{./images/ch_0-0-12.png}
        \caption{The convex hull of Fig.~\ref{fig:zero_raw}}
        \label{fig:zero_ch}
    \end{subfigure}
    \caption{An example of the convex hull transform of the digit zero}
    \label{fig:chull_example}
\end{figure}

The Property Inferencing phase of the architecture was implemented in three
different ways to compare performance of various NN models.  One implementation
used Multi-Layer Perceptrons (MLP), two hidden layers of 128 perceptrons.  The
second implementation used a two layer CNN.  The final implementation used
Resnet50.

Before using the explainable architecture, the MLP, CNN, and Resnet50 models were
trained and tested on untransformed data to establish a baseline for
unexplainable recognition.  The results are outlined in Table
\ref{tab_unexplainable_accuracy_results}.

After applying the explainable architecture to the balanced EMNIST dataset, a
significant degradation in accuracy was observed in comparison to the
unexplainable results as shown in Table~\ref{tab_explainable accuracy_results}.
After considering opportunities for improving performance, the option of further
splitting the EMNIST dataset by classes was used.  Balanced EMNIST was split
into a dataset consisting of only digits, only uppercase digits (26 classes),
and only lowercase letters.  Justification for such splitting involves
considering works that show there is improvement with recognition when context
of characters of a writing are considered \cite{506792}.  Based upon context, it
may be apparent from which split a letter belongs. After splitting the balanced
dataset, additional unexplainable and explainable baselines were obtained for
later comparison in Sec. \ref{emnist_results}.

\subsection{Explainability Metric for Unexplainable Components}
\label{method:metric}

\begin{equation}
    X_d=\frac{\sum_i x_{i,d}}{|V_d|}
    \label{eq:explainability}
\end{equation}

A strategy for improving the accuracy of the architecture, at the cost of
explainability, is to add a NN to the Property Inferencing stage that will be
trained on the untransformed input and treated as unexplainable in the system.
That unexplainable contribution would have a high effectiveness metric, because
of its accuracy, and would contribute to confidence but would detract from the
explanation.  Characterizing the explainability of a proposed solution from the
architecture may be accomplished by assigning an explainability component,
$x_{i,d}$, to each NN, $i$.  Explainable NNs would be assigned an explainability
component $1.0 \geq x_i > 0$ while unexplainable would be assigned $x_i = 0$.
Eq.~\ref{eq:explainability} gives explainability, $X_d$, for class $d$ where
$x_{i,d}$ represents the explainable component from NNs that vote for class $d$
and $V_d$ is the set of NNs that vote for class $d$.  Note that explainability,
$X_d$ would be $1.0 \geq X_d \geq 0$, with $X_d = 1.0$ signifying that only
explainable NNs voted for $d$ while values $X_d < 1.0$ would signify that a vote
was contributed by an unexplainable NN for class $d$.  In such a system, in
addition to the explainable rationale for decisions, the output would consist of
a set of triples representing the proposed decisions (class, confidence for the
class, and explainability metric for the class).

\subsection{Removing Ambiguous Data}
\label{method:pruning}

EMNIST is split in six different ways, based on classes included and the
balancing of classes in each split.  The split primarily used in this work is
the EMNIST balanced data set which contains 131,600 samples in 46 classes.

%consisting of decimal digits, 26 uppercase letters, and 10 lowercase letters.

%While perusing samples in the EMNIST the dataset there were observations of
%ambiguous and mislabeled samples.

Some of the label problems observed in EMNIST, from a cursory review, are shown in
Fig.~\ref{fig:emnist_samples}. Each row in the figure represents samples from
a class.  The first sample of each row appears to be acceptable,
subsequent samples in each row are not. One particular class, F, had a large
percentage, over 30\%, observed as lowercase f. Other classes were noted to have
ambiguous or overlapping symbols such as U and V; $0$, O, and D; and L, I, l and
the digit 1.

High incidence of label errors as well as ambiguous data causes accuracy
problems with ML and especially exacerbates XAI system results.  There was a
strong desire to address some of the issues.

\begin{figure}[h]
    \centering
    \textbf{Uppercase D Samples}\par\medskip 
    \begin{subfigure}{.10\textwidth}
        \centering
        \includegraphics[width=.90\textwidth]{./images/issues/D-0.png}
        %\caption{}
        \label{fig:issue_D01}
    \end{subfigure}%
    \begin{subfigure}{.10\textwidth}
        \centering
        \includegraphics[width=.90\textwidth]{./images/issues/D-01.png}
        %\caption{}
        \label{fig:issue_D01}
    \end{subfigure}%
    \begin{subfigure}{.10\textwidth}
        \centering
        \includegraphics[width=.90\textwidth]{./images/issues/D-02.png}
        %\caption{}
        \label{fig:issue_D02}
    \end{subfigure}%
    \begin{subfigure}{.10\textwidth}
        \centering
        \includegraphics[width=.90\textwidth]{./images/issues/D-03.png}
        %\caption{}
        \label{fig:issue_D03}
    \end{subfigure}%
    \begin{subfigure}{.10\textwidth}
        \centering
        \includegraphics[width=.90\textwidth]{./images/issues/D-04.png}
        %\caption{}
        \label{fig:issue_D04}
    \end{subfigure}\par\medskip
    \textbf{Uppercase F Samples}\par\medskip
    \begin{subfigure}{.10\textwidth}
        \centering
        \includegraphics[width=.90\textwidth]{./images/issues/F-0.png}
        %\caption{}
        \label{fig:issue_F0}
    \end{subfigure}%
    \begin{subfigure}{.10\textwidth}
        \centering
        \includegraphics[width=.90\textwidth]{./images/issues/F-01.png}
        %\caption{}
        \label{fig:issue_F01}
    \end{subfigure}%
    \begin{subfigure}{.10\textwidth}
        \centering
        \includegraphics[width=.90\textwidth]{./images/issues/F-02.png}
        %\caption{}
        \label{fig:issue_F02}
    \end{subfigure}%
    \begin{subfigure}{.10\textwidth}
        \centering
        \includegraphics[width=.90\textwidth]{./images/issues/F-03.png}
        %\caption{}
        \label{fig:issue_F03}
    \end{subfigure}%
    \begin{subfigure}{.10\textwidth}
        \centering
        \includegraphics[width=.90\textwidth]{./images/issues/F-04.png}
        %\caption{}
        \label{fig:issue_F04}
    \end{subfigure}\par\medskip
    \textbf{Uppercase H Samples}\par\medskip
    \begin{subfigure}{.10\textwidth}
        \centering
        \includegraphics[width=.90\textwidth]{./images/issues/H-0.png}
        %\caption{}
        \label{fig:issue_H0}
    \end{subfigure}%
    \begin{subfigure}{.10\textwidth}
        \centering
        \includegraphics[width=.90\textwidth]{./images/issues/H-02.png}
        %\caption{}
        \label{fig:issue_H01}
    \end{subfigure}%
    \begin{subfigure}{.10\textwidth}
        \centering
        \includegraphics[width=.90\textwidth]{./images/issues/H-03.png}
        %\caption{}
        \label{fig:issue_H02}
    \end{subfigure}%
    \begin{subfigure}{.10\textwidth}
        \centering
        \includegraphics[width=.90\textwidth]{./images/issues/H-04.png}
        %\caption{}
        \label{fig:issue_H03}
    \end{subfigure}%
    \begin{subfigure}{.10\textwidth}
        \centering
        \includegraphics[width=.90\textwidth]{./images/issues/H-05.png}
        %\caption{}
        \label{fig:issue_H04}
    \end{subfigure}\par\medskip
    \textbf{Uppercase N Samples}\par\medskip
    \begin{subfigure}{.10\textwidth}
        \centering
        \includegraphics[width=.90\textwidth]{./images/issues/N-0.png}
        %\caption{}
        \label{fig:issue_N0}
    \end{subfigure}%    
    \begin{subfigure}{.10\textwidth}
        \centering
        \includegraphics[width=.90\textwidth]{./images/issues/N-01.png}
        %\caption{}
        \label{fig:issue_N01}
    \end{subfigure}%
    \begin{subfigure}{.10\textwidth}
        \centering
        \includegraphics[width=.90\textwidth]{./images/issues/N-02.png}
        %\caption{}
        \label{fig:issue_N02}
    \end{subfigure}%
    \begin{subfigure}{.10\textwidth}
        \centering
        \includegraphics[width=.90\textwidth]{./images/issues/N-03.png}
        %\caption{}
        \label{fig:issue_N03}
    \end{subfigure}%
    \begin{subfigure}{.10\textwidth}
        \centering
        \includegraphics[width=.90\textwidth]{./images/issues/N-04.png}
        %\caption{}
        \label{fig:issue_N04}
    \end{subfigure}\par\medskip
    \textbf{Uppercase T Samples}\par\medskip
    \begin{subfigure}{.10\textwidth}
        \centering
        \includegraphics[width=.90\textwidth]{./images/issues/T-0.png}
        %\caption{}
        \label{fig:issue_T0}
    \end{subfigure}%    
    \begin{subfigure}{.10\textwidth}
        \centering
        \includegraphics[width=.90\textwidth]{./images/issues/T-01.png}
        %\caption{}
        \label{fig:issue_T01}
    \end{subfigure}%
    \begin{subfigure}{.10\textwidth}
        \centering
        \includegraphics[width=.90\textwidth]{./images/issues/T-02.png}
        %\caption{}
        \label{fig:issue_T02}
    \end{subfigure}%
    \begin{subfigure}{.10\textwidth}
        \centering
        \includegraphics[width=.90\textwidth]{./images/issues/T-03.png}
        %\caption{}
        \label{fig:issue_T03}
    \end{subfigure}%
    \begin{subfigure}{.10\textwidth}
        \centering
        \includegraphics[width=.90\textwidth]{./images/issues/T-04.png}
        %\caption{}
        \label{fig:issue_T04}
    \end{subfigure}
    \caption{EMNIST samples for the indicated uppercase letters.  Note that the first sample in each row appears as expected and others are ambiguous or mislabeled.}
    \label{fig:emnist_samples}
\end{figure}

In an attempt to improve explainable results, the removal of ambiguous and
mislabeled data was explored. Data and pruning of the training and test sets
were performed in several ways and compared.

The first pruning scheme was based on a threshold of 75\% confidence, which
eliminated 5\% of the training set. Results of threshold pruning were observed
on a per class basis and it was noted in Table \ref{tab_threshold_pruning_qty}
that in some classes, over half of the samples were removed.  E.g., the I and L
classes had more than 1200 of the 2400 samples for each class pruned. This
pruning caused a highly unbalanced training set and resulted in only 13\%
accuracy on uppercase letters.  Higher thresholds up to 95\% were also
attempted, however, poor accuracy results were observed.

\begin{table}
    \centering
    \caption{The number of pruned samples based on a 75\% confidence threshold}
    %\resizebox{\textwidth}{!}
    \label{tab_threshold_pruning_qty}
\end{table}

The second pruning scheme limited the pruning to only the worst 5\% confidence
of each class. Wrong classifications were especially penalized by assigning the
confidence a negative value, via multiplication by $-1.0$.  This scheme used a
K-fold Cross-validation to ensure that confidence of the training set samples
were obtained out of sample and without bias.

Another means of pruning involved leveraging Cleanlab to remove issues from the data.
Cleanlab was also used iteratively with three passes and results are reviewed in
\ref{emnist_results}.

\section{Results}
\label{results}

\subsection{MNIST Results}
\label{mnist_results}

Accuracy achieved previously with MNIST was 91.9\% with 100\% explainability on
handwritten digits\cite{whitten21}. Utilizing an unexplainable component,
consisting of the original image, and an additional convex hull transform
resulted in an improvement to 97.7\% accuracy.  The resulting average
explainability metric, $X_d$, was reduced to 79\%.

%Results for addition of the convex hull are 97.1\% accuracy.  TODO add more details
%here.

%Results for adding the two hidden layer CNN are 96.0\% versus 91.9, a 4\%
%increase on the MNIST dataset from \cite{whitten21}.  TODO: add more here.

\subsection{EMNIST Results}
\label{emnist_results}

Table~\ref{tab_unexplainable_accuracy_results} shows the accuracy results using
the indicated NN on the various EMNIST dataset and splits. On the full balanced
dataset, with 46 classes, Resnet50 and the CNN performed comparably. The two
layer CNN can be observed as having maximum accuracy results in all cases except
lowercase characters.  The reason for lower Resnet results may be due to
resizing the input images since the Resnet implementation that was utilized
requires a minimum input of 32x32.  

\begin{table}
    \centering
    \caption{Unexplainable accuracy results on various balanced EMNIST data sets with differing NN models}
    \resizebox{\columnwidth}{!}{%
    \begin{tabular}{|l|c|c|c|c|c|c|c|c|}
        \cline{2-9}
        \multicolumn{1}{c}{} & \multicolumn{8}{|c|}{Accuracy (\%)} \\
        \cline{2-9}
        \multicolumn{1}{c}{} & \multicolumn{2}{|c|}{Full EMNIST} & \multicolumn{2}{|c|}{EMNIST Digits} & \multicolumn{2}{|c|}{EMNIST Caps} & \multicolumn{2}{|c|}{EMNIST Lower} \\
        \hline
        ML Model & Train & Test & Train & Test & Train & Test & Train & Test \\
        \hline
        \hline
        MLP & 82.59 & 78.68 & 99.82 & 97.32 & 95.88 & 91.13 & 97.91 & 89.47 \\
        \hline
        Two layer CNN & 89.14 & 88.66 & 99.44 & 99.05 & 95.87 & 95.36 & 94.84 & 93.41 \\
        \hline
        Resnet50 & 88.69 & 88.68 & 99.44 & 98.90 & 95.26 & 94.89 & 97.07 & 92.68 \\
        \hline
    \end{tabular}
    }
    \label{tab_unexplainable_accuracy_results}
\end{table}

Table \ref{tab_explainable accuracy_results} shows the explainable accuracy
results on the unpruned data. Compared to Table
\ref{tab_unexplainable_accuracy_results} the results are lower.  As expected,
there appears to be a cost explainability.  That cost is decreased accuracy.
%TODO: add more details here.

In the explainable results, Table~\ref{tab_explainable accuracy_results}, the
Resnet50 models appeared to perform better with the full EMNIST dataset and
uppercase characters.

\begin{table}
    \centering
    \caption{Explainable accuracy results on various balanced EMNIST data sets with differing NN models}
    \resizebox{\columnwidth}{!}{%
    \begin{tabular}{|l|c|c|c|c|c|c|c|c|}
        \cline{2-9}
        \multicolumn{1}{c}{} & \multicolumn{8}{|c|}{Accuracy (\%)} \\
        \cline{2-9}
        \multicolumn{1}{c}{} & \multicolumn{2}{|c|}{Full EMNIST} & \multicolumn{2}{|c|}{EMNIST Digits} & \multicolumn{2}{|c|}{EMNIST Caps} & \multicolumn{2}{|c|}{EMNIST Lower} \\
        \hline
        ML Model & Train & Test & Train & Test & Train & Test & Train & Test \\
        \hline
        \hline
        MLP & 71.04 & 64.99 & 95.10 & 90.80 & 81.22 & 73.81 & 91.68 & 81.41 \\
        \hline
        Two layer CNN & 68.21 & 64.97 & 93.63 & 92.65 & 84.93 & 82.88 & 88.77 & 86.57 \\
        \hline
        Resnet50 & 87.06 & 73.93 & 93.10 & 91.86 & 90.79 & 86.75 & 79.10 & 81.30 \\
        \hline
    \end{tabular}
    }
    \label{tab_explainable accuracy_results}
\end{table}

Table~\ref{raw_cap_confusion_matrix} shows the confusion matrix for interesting
uppercase letters after running though the explainable architecture. There is
significant confusion with the I and L classes and a fair degree of confusion
with the U and V classes.

%\begin{figure}[h]
%    \centering
%    \includegraphics[width=10cm]{./images/confusion_matrix_vlsid.png}
%    \caption{Confusion matrix using convex hull}
%    \label{fig:conf_matrix}
%\end{figure}

\begin{table}
    \centering
    \caption{Explainable confusion matrix for unpruned uppercase letters H - M, U, and V}
    %\resizebox{\columnwidth}{!}{%
    \begin{tabular}{ |c|c|c|c|c|c|c|c|c|}
    \hline
     & H & I & J & K & L & M & U & V \\
    \hline
    H & 374 &  & 1 & 3 &  & 11 & 1 & 1 \\
    \hline
    I &  & 275 & 2 & & 114 & & & \\
    \hline
    J &  & 11 & 376 & & 1 & & 2 & 6 \\
    \hline
    K &  &  &  & 392 & 1 & & & \\
    \hline
    L &  & 90 &  &  & 306 & & & \\
    \hline
    M &  &  &  &  &  & 399 & & 1 \\
    \hline
    U & 4 & 1 & & 1 & 9 & & 305 & 29 \\
    \hline
    V & & & 2 & & 2 & 2 & 9 & 326 \\
    \hline
    \end{tabular}
    %}
    \label{raw_cap_confusion_matrix}
\end{table}

After applying one round of Cleanlab to prune only training labels with issues,
Table \ref{raw_cap_cleanlab_confusion_matrix} indicates the same uppercase
letters of interest. The confusion in the I class appears to have gotten better.
L, U, and V classes appear to have gotten worse. Six of the eight classes of
interest had decreases in correct classifications, the values along the
diagonal.  Using CNNs in the explainable architecture, the overall results of
one round of Cleanlab were an increase in accuracy to 87.4\% on uppercase
letters from 82.9\% without pruning.

\begin{table}
    \centering
    \caption{Explainable confusion matrix for uppercase letters H - M, U, and V after a single Cleanlab pruning}
    %\resizebox{\columnwidth}{!}{%
    \begin{tabular}{ |c|c|c|c|c|c|c|c|c|}
    \hline
    ~ & H & I & J & K & L & M & U & V \\
    \hline
    H & 368 & & 1 & 2 & & 7 & 1 & \\
    \hline
    I &  & 293 & 4 &  & 65 & 14 & & \\
    \hline
    J & & 17 & 339 & & 1 & 6 & 1 & 2 \\
    \hline
    K & 5 &  &  & 349 & 2 & 20 &  & \\
    \hline
    L & 0 & 120 &  & 1 & 236 & 2 & & \\
    \hline
    M & 4 & & & 3 & & 383 & & \\
    \hline
    U & 2 & & 1 & 1 & & 22 & 304 & 17 \\
    \hline
    V & & & 3 & 1 & & 21 & 13 & 315 \\
    \hline
    \end{tabular}
    %}
    \label{raw_cap_cleanlab_confusion_matrix}
\end{table}

After applying subsequent rounds of Cleanlab to the capital letter dataset, the
accuracy achieved was 86.6\% after the second and 88.7\% after the third
iteration of pruning. Table \ref{raw_cap_cleanlab_third_confusion_matrix} again
shows the confusion matrix for the classes of interest after the third Cleanlab
pruning.

The best pruning results were achieved performing the 5\% pruning with a penalty
for incorrect labels.  The results on this pruned set were about 89.2\% accuracy
on uppercase letters versus 82.9\% on unpruned test data using CNN in the
explainable architecture.  

\begin{table}
    \centering
    \caption{Explainable confusion matrix for uppercase letters H - M, U, and V after three rounds of Cleanlab pruning}
    %\resizebox{\columnwidth}{!}{%
    \begin{tabular}{ |c|c|c|c|c|c|c|c|c|}
    \hline
    ~ & H & I & J & K & L & M & U & V \\
    \hline
    H & 357 & &  & 8 & & 2 & 1 & 1 \\
    \hline
    I &  & 254 & 3 & 3 & 101 & 1 & & \\
    \hline
    J & & 19 & 298 & 1 & 16 & & 2 & 6 \\
    \hline
    K & 3 &  &  & 352 & 2 & 2 & 0 & \\
    \hline
    L & 0 & 90 & 1 & 1 & 277 & & & \\
    \hline
    M & & & & 3 & & 378 & & 1 \\
    \hline
    U & 1 & & 1 & 9 & & 1 & 305 & 29 \\
    \hline
    V & & 2 & & 2 & 2 & 1 & 9 & 326 \\
    \hline
    \end{tabular}
    %}
    \label{raw_cap_cleanlab_third_confusion_matrix}
\end{table}


\subsection{Explainability Results}
\label{results:exp}

%We can show that since we are using the probabilistic voting that explainability quality is 100\%.
%Can we introduce other metrics?

This section depicts the explainability results using examples from the EMNIST
balanced test set using only explainable properties.  Example one, which is
labeled as an uppercase Q, is shown in Fig.~\ref{fig:ex1}.  Table
\ref{table:example1} shows that votes for classes Q, R, C and D were made by
various properties.  Six of the properties contributed a vote for Q and one
property each voted on the other classes.  Summing the effectiveness weights for
the classes, Q wins with a confidence of 77\%.

\begin{figure}
    \centering
    \includegraphics[width=1.5cm]{./images/examples/test-Q-3.png}
    \caption{Example 1: a test sample labeled Q}
    \label{fig:ex1}
\end{figure}

\begin{table}
    \caption{Prob. voting, effectiveness, and explainability for example 1}
    \centering
    \resizebox{\columnwidth}{!}{%
    \begin{tabular}{| c | c | c | c | c | c | c | c | c | c | c |}
    \cline{4-11}
    \multicolumn{3}{c}{} & \multicolumn{4}{|c|}{Effectiveness} & \multicolumn{4}{c|}{Explainability} \\
    \hline
     $P_j$ & Property & Vote & $E_{j,C}$ & $E_{j,D}$ & $E_{j,Q}$ & $E_{j,R}$ & $X_C$ & $X_D$ & $X_Q$ & $X_R$ \\
    \hline \cline{0-10}
    $P_0$ & Stroke & Q &  &  & 0.94 &  &  &  & \checkmark & \\ 
    \hline
    $P_1$ & Circle & C & 0.39 &  &  &  & \checkmark &  &  &  \\
    \hline
    $P_2$ & Crossing & Q &  &  & 0.71 &  &  &  & \checkmark &  \\
    \hline
    $P_3$ & Circle & D & & 0.21 &  &  &  & \checkmark &  &  \\
    \hline
    $P_4$ & Ellipse & R &  &  &  & 0.90 &  &  &  & \checkmark \\
    \hline
    $P_5$ & Endpoint & Q &  &  & 0.83 &  &  &  & \checkmark &  \\
    \hline
    $P_6$ & Encl. Reg. & Q &  &  & 0.85 &  &  &  & \checkmark &  \\
    \hline
    $P_7$ & Line & Q &  &  & 0.63 &  &  &  & \checkmark &  \\
    \hline
    $P_8$ & Encl. Reg. &  &  &  &  &  &  &  &  &  \\
    \hline
    $P_9$ & Area & Q &  &  & 0.95 &  &  &  & \checkmark &  \\
    \hline \cline{0-10}
    \multicolumn{3}{|c|}{Weight} & 0.39 & 0.21 & 4.91 & 0.90 & \multicolumn{4}{c|}{$\sum W_k=6.41$} \\
    \cline{0-10}
    \multicolumn{3}{|c|}{Confidence} & $6\%$ & $3\%$ & $77\%$ & $14\%$ & \multicolumn{4}{c}{} \\
    \cline{0-6}
    \end{tabular}%
    }
    \label{table:example1}
\end{table}

Explainable rationale provided by XAI is depicted in Table~\ref{table:exexample1} for the four classes that
received votes from the NNs.

\begin{table}
    \caption{Rationale for example 1}
    \centering
    \begin{tabular}{| p{0.04\linewidth} | p{0.14\linewidth} | p{0.65\linewidth} |}
    \hline
     $X_d$ & Confidence & Explainable Description \\
    \hline \cline{1-3}
    $X_Q$ & 77\% & Confidence is high for interpreting as a Q due to the stroke, area, crossing, line, and enclosed region properties. \\ 
    \hline
    $X_R$ & 14\% & Confidence is low for interpreting as an R due to the ellipse property. \\
    \hline
    $X_C$ & 6\% & Confidence is low for interpreting as a C due to the circle property. \\
    \hline
    $X_D$ & 3\% & Confidence is low for interpreting as a D due to the circle property. \\
    \hline
    \end{tabular}
    \label{table:exexample1}
\end{table}

Figure \ref{fig:ex2} shows example two from the test set which is labeled as an
uppercase C. Votes by the properties in Table \ref{table:example2} show that
several properties infer an uppercase C.  The overall weight for the class C is
4.97 from the effectiveness of the properties giving it a confidence of 76\%.
The uppercase Z receives a vote from the enclosed region property. Uppercase O
receives a vote from the circle property via the ellipse transform.  The class C
wins with 76\% confidence. Table \ref{table:exexample2} shows the explainability
for Example 2.

\begin{figure}
    \centering
    \includegraphics[width=1.5cm]{./images/examples/test-C-1.png}
    \caption{Example 2: A test sample labeled C}
    \label{fig:ex2}
\end{figure}

\begin{table}
    \caption{Prob. voting, effectiveness, and explainability for example 2}
    \centering
    \resizebox{\columnwidth}{!}{%
    \begin{tabular}{| c | c | c | c | c | c | c | c | c |}
    \cline{4-9}
    \multicolumn{3}{c}{} & \multicolumn{3}{|c|}{Effectiveness} & \multicolumn{3}{c|}{Explainability} \\
    \hline
     $P_j$ & Property & Vote & $E_{j,C}$ & $E_{j,O}$ & $E_{j,Z}$ & $X_C$ & $X_O$ & $X_Z$ \\
    \hline \cline{0-8}
    $P_0$ & Stroke & C & 0.91 &  &  & \checkmark &  & \\ 
    \hline
    $P_1$ & Circle & C & 0.32 &  &  & \checkmark &  &  \\
    \hline
    $P_2$ & Crossing & C & 0.94 &  &  & \checkmark &  &  \\
    \hline
    $P_3$ & Circle & C & 0.39 &  &  & \checkmark &  &  \\
    \hline
    $P_4$ & Ellipse & O &  & 0.61 &  &  & \checkmark &  \\
    \hline
    $P_5$ & Endpoint & C & 0.82 &  &  & \checkmark &  &  \\
    \hline
    $P_6$ & Encl. Reg. & Z &  &  & 0.98 &  &  & \checkmark \\
    \hline
    $P_7$ & Line & C & 0.69 &  &  & \checkmark &  &  \\
    \hline
    $P_8$ & Encl. Reg. &  &  &  &  &  &  &  \\
    \hline
    $P_9$ & Area & C & 0.89 &  &  & \checkmark &  &  \\
    \hline \cline{0-8}
    \multicolumn{3}{|c|}{Weight} & 4.97 & 0.61 & 0.98 & \multicolumn{3}{c|}{$\sum W_k=6.56$} \\
    \cline{0-8}
    \multicolumn{3}{|c|}{Confidence} & $76\%$ & $9\%$ & $15\%$ & \multicolumn{3}{c}{} \\
    \cline{0-5}
    \end{tabular}%
    }
    \label{table:example2}
\end{table}

\begin{table}
    \caption{Rationale for example 2}
    \centering
    \begin{tabular}{| p{0.04\linewidth} | p{0.14\linewidth} | p{0.65\linewidth} |}
    \hline
     $X_d$ & Confidence & Explainable Description \\
    \hline \cline{1-3}
    $X_C$ & 76\% & Confidence is high for interpreting as a C due to the crossing, stroke, area, endpoint, line, and circle properties. \\ 
    \hline
    $X_Z$ & 15\% & Confidence is low for interpreting as a Z due to the enclosed region property. \\
    \hline
    $X_O$ & 9\% & Confidence is low for interpreting as a O due to the ellipse property. \\
    \hline
    \end{tabular}
    \label{table:exexample2}
\end{table}

The vote for an uppercase Z, by the enclosed region, is unexpected
as there should be no resulting enclosed pixels in the transformed C,
i.e., a flood fill, which is the transform associated with the enclosed region,
should result in no pixels in the output image since a C has no enclosed region.
A review of the resulting transform shows that this is indeed the case.  Test
results show that samples form several classes, such as Y and L, that would not
have pixels in a flood fill, get Z votes from the enclosed region property.  This
suggests that a future improvement in the system could be to discount results
from transforms void of any activated pixels, especially when the transform has
several classes that generate empty images.  Improvements could also be
accomplished using knowledge of the explainable properties associated and not
associated with each class.

%Table \ref{table:exexample2} shows the explainability for Example 2.  In this case, the system indicates:

% \begin{itemize}
%     \item Confidence is high for interpreting as a C due to the stroke, endpoint, area, circle, and crossing properties.
%     \item Confidence is low for interpreting as a Z due to enclosed region properties.
%     \item Confidence is low for interpreting as a O due to circle properties.
% \end{itemize}

\subsection{Resolving Mislabeled Pruned Data}
\label{results:resolving}

Images pruned from the training set due to issues such as mislabeled or
ambiguous samples after one Cleanlab pruning were submitted back to the system.  
Of the 1188 pruned images only 129, or about 11\%, of the problematic labels from
the original training set were recognized as correctly labeled.  The remaining
89\% were recognized by the explainable architecture as other classes with
varying degrees of confidence. Some of the pruned images are depicted in
Fig.~\ref{fig:pruned_inf_samples}.  In many cases the explainable architecture is
able to correctly inference, according to human recognition, mislabeled samples.

\begin{figure}[h]
    \centering
    %\textbf{Pruned Samples}\par\medskip 
    \begin{subfigure}{.20\columnwidth}
        \centering
        \includegraphics[width=.90\textwidth]{./images/issues/excluded-G-0-inferred-S.png}
        \caption{G label}
        \label{fig:inf_issue_G0}
    \end{subfigure}%
    \begin{subfigure}{.20\columnwidth}
        \centering
        \includegraphics[width=.90\textwidth]{./images/issues/excluded-T-5-inferred-u.png}
        \caption{T label}
        \label{fig:inf_issue_T5}
    \end{subfigure}%
    \begin{subfigure}{.20\columnwidth}
        \centering
        \includegraphics[width=.90\textwidth]{./images/issues/excluded-O-7-inferred-M.png}
        \caption{O label}
        \label{fig:inf_issue_O7}
    \end{subfigure}%
    \begin{subfigure}{.20\columnwidth}
        \centering
        \includegraphics[width=.90\textwidth]{./images/issues/excluded-W-3-inferred-M.png}
        \caption{W label}
        \label{fig:inf_issue_W3}
    \end{subfigure}%
    %\begin{subfigure}{.20\columnwidth}
    %    \centering
    %    \includegraphics[width=.90\textwidth]{./images/issues/excluded-O-10-inferred-F.png}
    %    \caption{O label}
    %    \label{fig:inf_issue_O10}
    %\end{subfigure}
    \caption{Pruned samples that were inferred by the system.}
    \label{fig:pruned_inf_samples}
\end{figure}

\begin{figure}[h]
    \centering
    %\textbf{Pruned Samples}\par\medskip 
    \begin{subfigure}{.20\columnwidth}
        \centering
        \includegraphics[width=.90\textwidth]{./images/issues/excluded-G-0-stroke.png}
        \caption{Stroke}
        %\label{fig:inf_issue_G0_stroke}
    \end{subfigure}%
    \begin{subfigure}{.20\columnwidth}
        \centering
        \includegraphics[width=.90\textwidth]{./images/issues/excluded-G-0-circle.png}
        \caption{Circles}
        %\label{fig:inf_issue_T5}
    \end{subfigure}%
    \begin{subfigure}{.20\columnwidth}
        \centering
        \includegraphics[width=.90\textwidth]{./images/issues/excluded-G-0-el-circle.png}
        \caption{Ellipse}
        %\label{fig:inf_issue_O7}
    \end{subfigure}%
    \begin{subfigure}{.20\columnwidth}
        \centering
        \includegraphics[width=.90\textwidth]{./images/issues/excluded-G-0-area.png}
        \caption{Area}
        %\label{fig:inf_issue_W3}
    \end{subfigure}%
    \begin{subfigure}{.20\columnwidth}
        \centering
        \includegraphics[width=.90\textwidth]{./images/issues/excluded-G-0-line.png}
        \caption{Line}
        %\label{fig:inf_issue_O10}
    \end{subfigure}
    \caption{Relevant transforms of Fig.~\ref{fig:inf_issue_G0}.}
    \label{fig:g0_trans}
\end{figure}

Fig.~\ref{fig:inf_issue_G0}, originally labeled as a G, was inferred correctly by
the explainable architecture as an S with 42\% confidence.  The next closest
prediction was a Z with 14\% confidence.  The explanation for the prediction was
due to the stroke, circle, area, and line properties being consistent with the
letter S. Examples of the contributing transforms are depicted in
Fig.~\ref{fig:g0_trans}.

\begin{figure}[h]
    \centering
    %\textbf{Pruned Samples}\par\medskip 
    \begin{subfigure}{.20\columnwidth}
        \centering
        \includegraphics[width=.90\textwidth]{./images/issues/excluded-T-5-stroke.png}
        \caption{Stroke}
        %\label{fig:inf_issue_G0_stroke}
    \end{subfigure}%
    \begin{subfigure}{.20\columnwidth}
        \centering
        \includegraphics[width=.90\textwidth]{./images/issues/excluded-T-5-area.png}
        \caption{Area}
        %\label{fig:inf_issue_T5}
    \end{subfigure}%
    \begin{subfigure}{.20\columnwidth}
        \centering
        \includegraphics[width=.90\textwidth]{./images/issues/excluded-T-5-endpoint.png}
        \caption{Endpoint}
        %\label{fig:inf_issue_O7}
    \end{subfigure}%
    \begin{subfigure}{.20\columnwidth}
        \centering
        \includegraphics[width=.90\textwidth]{./images/issues/excluded-T-5-circle.png}
        \caption{Circle}
        %\label{fig:inf_issue_W3}
    \end{subfigure}%
    %\begin{subfigure}{.20\columnwidth}
    %    \centering
    %    \includegraphics[width=.90\textwidth]{./images/issues/excluded-G-0-line.png}
    %    \caption{Line}
    %    %\label{fig:inf_issue_O10}
    %\end{subfigure}
    \caption{Relevant transforms of Fig.~\ref{fig:inf_issue_T5}.}
    \label{fig:t5_trans}
\end{figure}

Fig.~\ref{fig:inf_issue_T5}, originally labeled as a T, was inferred correctly as
a U with 36\% confidence. The next closest prediction was a Z with 25\%
confidence.  The explanation was due to the stroke, area,
endpoint, and circle properties. Fig.~\ref{fig:t5_trans}
depicts relevant transforms.

\begin{figure}[h]
    \centering
    %\textbf{Pruned Samples}\par\medskip 
    \begin{subfigure}{.20\columnwidth}
        \centering
        \includegraphics[width=.90\textwidth]{./images/issues/excluded-O-7-stroke.png}
        \caption{Stroke}
        %\label{fig:inf_issue_G0_stroke}
    \end{subfigure}%
    \begin{subfigure}{.20\columnwidth}
        \centering
        \includegraphics[width=.90\textwidth]{./images/issues/excluded-O-7-endpoint.png}
        \caption{Endpoint}
        %\label{fig:inf_issue_T5}
    \end{subfigure}%
    \begin{subfigure}{.20\columnwidth}
        \centering
        \includegraphics[width=.90\textwidth]{./images/issues/excluded-O-7-area.png}
        \caption{Area}
        %\label{fig:inf_issue_O7}
    \end{subfigure}%
    \begin{subfigure}{.20\columnwidth}
        \centering
        \includegraphics[width=.90\textwidth]{./images/issues/excluded-O-7-circle.png}
        \caption{Circle}
        %\label{fig:inf_issue_W3}
    \end{subfigure}%
    %\begin{subfigure}{.20\columnwidth}
    %    \centering
    %    \includegraphics[width=.90\textwidth]{./images/issues/excluded-O-7-fill.png}
    %    \caption{Enclosed Region}
    %    %\label{fig:inf_issue_O10}
    %\end{subfigure}
    \caption{Relevant transforms of Fig.~\ref{fig:inf_issue_O7}.}
    \label{fig:o7_trans}
\end{figure}

Fig.~\ref{fig:inf_issue_O7}, originally labeled as an O, was inferred correctly
as an M with 56\% confidence. The next closest prediction was a Z with 27\%
confidence.  Rationale for the decision are the stroke, endpoint, area,
and circle being consistent with an M. Fig.~\ref{fig:o7_trans} depicts the
transforms that contributed to the decision.

\begin{figure}[h]
    \centering
    %\textbf{Pruned Samples}\par\medskip 
    \begin{subfigure}{.20\columnwidth}
        \centering
        \includegraphics[width=.90\textwidth]{./images/issues/excluded-W-3-stroke.png}
        \caption{Stroke}
        %\label{fig:inf_issue_G0_stroke}
    \end{subfigure}%
    \begin{subfigure}{.20\columnwidth}
        \centering
        \includegraphics[width=.90\textwidth]{./images/issues/excluded-W-3-endpoint.png}
        \caption{Endpoint}
        %\label{fig:inf_issue_T5}
    \end{subfigure}%
    \begin{subfigure}{.20\columnwidth}
        \centering
        \includegraphics[width=.90\textwidth]{./images/issues/excluded-W-3-area.png}
        \caption{Area}
        %\label{fig:inf_issue_O7}
    \end{subfigure}%
    \begin{subfigure}{.20\columnwidth}
        \centering
        \includegraphics[width=.90\textwidth]{./images/issues/excluded-W-3-line.png}
        \caption{Line}
        %\label{fig:inf_issue_W3}
    \end{subfigure}%
    %\begin{subfigure}{.20\columnwidth}
    %    \centering
    %    \includegraphics[width=.90\textwidth]{./images/issues/excluded-O-7-fill.png}
    %    \caption{Enclosed Region}
    %    %\label{fig:inf_issue_O10}
    %\end{subfigure}
    \caption{Relevant transforms of Fig.~\ref{fig:inf_issue_W3}.}
    \label{fig:w3_trans}
\end{figure}

Fig.~\ref{fig:inf_issue_W3}, originally labeled as a W, was inferred correctly
as an M with 65\% confidence. The next closest character was a Z with 28\%
confidence.  Rationale for selecting M was consistency with the stroke,
endpoint, area, and line properties.  Relevant transforms are depicted in
Fig.~\ref{fig:w3_trans}

%Fig.~\ref{fig:inf_issue_O10}, originally labeled as an O was inferred correctly as an F with 40\% confidence.

The quality of the training set is a big concern with NN and XAI.  The pruning
methods presented attempt to improve the quality of the training set, to enhance
accuracy and explainability of the system. With respect to pruned samples, the
explainable architecture was a benefit in two ways, (a) it was able to correct
some problematic, pruned labels with a good degree of confidence (b) the
explainable architecture provided rationale based on explainable properties that
could help a user to confidently resolve issues with labels.

\section{Conclusion}
\label{conclusion}

%Achieved acceptable accuracy results.  Explainability based on properties and trust in the system.

%Cost in accuracy for explainability.

%The XAI pipeline was able to inference mislabeled samples.

%TODO: add more.

%This paper introduces a neural network architecture which recognizes handwritten characters and produces explanations based on a character property analysis.

This work introduced an explainable architecture for recognizing handwritten
characters.  This work increased the accuracy of the architecture by leveraging
multiple NN architectures and by adding highly accurate but unexplainable
components. A metric is introduced that assists in characterizing explainability
when unexplainable components are added to the Architecture.

The expansive EMNIST dataset was studied for explainable recognition of
characters and some challenges with training data were identified.  The dataset
used consists of about five times the number labels of previous data sets. There
was a degree of ambiguity between some classes. Some of the ambiguity problems
were avoided by further splitting the dataset.

This work introduces the concept of correctly pruning a training set in order
to avoid misleading training set labels and improve the explainability. Results
processing the EMNIST character database using the explainable architecture are
provided. The explainable architecture was also used to resolve some of the
training set issues, leveraging explanations to increase confidence of new
labels.

Further investigations for potential improvements to the explainable
architecture involve exploring varying NN architectures as well as ensuring that
transforms absent of activated pixels do not contribute to decisions. 

\bibliographystyle{IEEEtran}
\bibliography{references}

\end{document}
